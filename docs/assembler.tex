\documentclass[a4paper]{report}

\usepackage[british]{babel}

\title{A Level Computer Science NEA}
\author{Bailey Harrison}
%\date{January 1 1984}

\begin{document}

\maketitle

\tableofcontents



\chapter{Analysis}

\section{Introduction}

An assembler is a computer program that translates low-level assembly instructions
into raw machine code and data. Unlike compilers and interpreters, assemblers lack
many of the high-level control statements that most programmers are used to, such
as 'if' statements and 'for' loops.

Generally, there is a 1:1 correspondence between assembler opcodes (an acronym
specifying the operation) and machine code instructions. Assemblers also support a
few other features, notably labels. A label serves as a placeholder for a memory
address, which can be used to implement variables and functions in your program.

\section{Problem}

\section{System Objectives / Specification}



\chapter{Design}



\chapter{Technical Solution}



\chapter{Testing}



\chapter{Maintenance}



\chapter{User Manual}



\chapter{Appraisal}



\end{document}
