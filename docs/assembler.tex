\documentclass[a4paper]{report}

\usepackage[british]{babel}

\usepackage{listings}
\usepackage{color}

\definecolor{dkgreen}{rgb}{0,0.6,0}
\definecolor{gray}{rgb}{0.5,0.5,0.5}
\definecolor{mauve}{rgb}{0.58,0,0.82}

\lstset{frame=tb,
  language=C,
  aboveskip=3mm,
  belowskip=3mm,
  showstringspaces=false,
  columns=flexible,
  basicstyle={\small\ttfamily},
  numbers=none,
  numberstyle=\tiny\color{gray},
  keywordstyle=\color{blue},
  commentstyle=\color{dkgreen},
  stringstyle=\color{mauve},
  breaklines=true,
  breakatwhitespace=true,
  tabsize=4
}

\title{A Level Computer Science NEA}
\author{Bailey Harrison}
%\date{January 1 1984}

\begin{document}

\maketitle

\tableofcontents



\chapter{Analysis}

\section{Introduction}

An assembler is a computer program that translates low-level assembly instructions
into raw machine code and data. Unlike compilers and interpreters, assemblers lack
many of the high-level control statements that most programmers are used to, such
as 'if' statements and 'for' loops.

Generally, there is a 1:1 correspondence between assembler opcodes (an acronym
specifying the operation) and machine code instructions (the binary data that a CPU
decodes and executes). Assemblers also support a few other features, notably labels.
A label serves as a placeholder for a memory address, which can be used to implement
variables and functions in a program.

\section{Problem}

\section{System Objectives / Specification}



\chapter{Design}



\chapter{Technical Solution}

\section{Source Files}

\subsection{main.c}
\lstinputlisting{../src/main.c}
\subsection{symtable.c}
\lstinputlisting{../src/symtable.c}
\subsection{assemble.c}
\lstinputlisting{../src/assemble.c}
\subsection{parseline.c}
\lstinputlisting{../src/parseline.c}
\subsection{util.c}
\lstinputlisting{../src/util.c}

\section{Header Files}

\subsection{symtable.h}
\lstinputlisting{../src/symtable.h}
\subsection{assemble.h}
\lstinputlisting{../src/assemble.h}
\subsection{parseline.h}
\lstinputlisting{../src/parseline.h}
\subsection{util.h}
\lstinputlisting{../src/util.h}



\chapter{Testing}



\chapter{Maintenance}



\chapter{User Manual}



\chapter{Appraisal}



\end{document}
